\section{The web inteface VELS\_WEB} \label{VELS_WEB}

The configuration and status web interface VELS\_WEB was designed as an easy interface
for a course operator to view and change course parameters and monitor the course progress.
It is implemented with web2py and can be run as a daemon accessing the datastores of the
autosub submission system. This web interface was only 
designed for internal operator use! Therefore it should only be reachable via 
the internal networks using a web browser.

The menu items of the VELS\_WEB are the following:
\begin{description}
\item [Start:] The start page.
\item [Users:] View of the registered students, allows to change or a student via the
    "Edit" button, delete a student via the "Delete" button  and show the progress of
    a student in the course via the "View" button. The table is sortable by column by
    clicking on the head of the column.
\item [Tasks:] The task configuration, will be discussed in Section \ref{sub:configTasks}.
\item [Whitelist:] Change whitelisting for students, will be discussed in Section
    \ref{sub:whitelisting}.
\item [General Config:] Change dynamic configuration items of VELS, will be discussed in
    Section \ref{sub:generalconfig}.
\item [Statistics:] View statistics about sent and received emails and task success of
    the students.
\item [User Tasks:] View mappings from students to individual tasks.
\end{description}

Usage of the VELS\_WEB interface will be discussed with system setup in Section \ref{system_setup}.
