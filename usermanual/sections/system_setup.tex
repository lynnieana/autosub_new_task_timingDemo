\section{Setting up the VELS system} \label{system_setup}

\subsection{VELS server setup} \label{sub:serversetup}

The first thing to do when you want to install the autosub submission system and the VELS
web interface is to clone the git repository:

\begin{verbatim}
git clone https://github.com/autosub-team/autosub.git
\end{verbatim}

\subsection{Configuration File Creation} \label{sub:exampleconfig}
The autosub config file is used to configure the connection to the mail server and set
configuration parameters for the system. In order to run the autosub daemon a config file
has to be created. The config file fields each belong to one of the groups embraced 
in [...]. An example config file can be seen below, the full list of fields and their
meaning is described in Appendix 
Section \ref{app:config}

\begin{lstlisting}[frame=single,captionpos=b,caption=example.cfg, belowcaptionskip=4pt]]
[imapserver]
servername: imap.gmail.com
serverport: 993
security: ssl
username: submission@gmail.com
password: mysupersecurepassword
email: submission@gmail.com

[smtpserver]
servername: smtp.gmail.com
serverport: 587
security: starttls
username: submission@gmail.com
password: mysupersecurepassword
email: submission@gmail.com

[system]
num_workers: 20
queue_size: 200
poll_period: 5
log_dir: /home/martin/autosub/src
log_threshhold: INFO

[course]
tasks_dir: /home/martin/autosub/tasks/implementation/VHDL
course_name: My Cool Course
mode: normal
allow_skipping: no
auto_advance: no
\end{lstlisting}

\subsection{Starting the Daemon}

Autosub is implemented as a daemon process, that means all messages provided are written
to files (see Section \ref{logerror} and \ref{tasklog} for details on those files) -- 
nothing is written to the console. The daemon is started using a shell-script located
in {\tt autosub/src/}:

\begin{verbatim}
./autosub.sh start
\end{verbatim}

This starts the daemon using the default configuration file named {\tt default.cfg}. If you
wan to use your own configuration file, you have to pass the files name to the scrip when
starting the daemon:

\begin{verbatim}
./autosub.sh start myconfig.cfg
\end{verbatim}

To stop the daemon just run the command:

\begin{verbatim}
./autosub.sh stop
\end{verbatim}

\subsection{Setting up the VELS\_WEB Configuration Interface}
To use the VELS\_WEB Configuration Interface it has first to be installed and
configured using these steps:

Change to the VELS\_WEB directory.

\begin{verbatim}
python3 installer.py <pathtoautosub> <pathtoconfigfile>
\end{verbatim}

Use the same configfile you used for starting the autosub daemon! This step will
also download web2py to the user's home folder and set needed symbolic links to 
connect VELS\_WEB to the autosub system.

If you change parameters in the category $[$system$]$ in your autosub config file or need to switch to 
another config file run the installer with the reinstall flag:

\begin{verbatim}
python3 installer.py --reinstall <pathtoautosub> <pathtoconfigfile>
\end{verbatim}

To use https you have to use a SSL key. The system expects the keyfiles
(server.crt , server.csr , server.key) to be in the web2py directory. To
generate keys and place them run the following (this can be skipped if you 
already have keyfiles you can use!):

\begin{verbatim}
./genkeys.sh    
\end{verbatim}

To start the VELS\_WEB daemon at port <port> and with web2py admin password 
<password> (this can be set by you!) run:
\begin{verbatim}
./daemon.sh start <port> <password> 
\end{verbatim}

After this step the VELS\_WEB interface will be reachable via your browser at
address:
\begin{verbatim}
https://<server_ip>:<port>/VELS_WEB
\end{verbatim}

To stop the daemon just run the command:

\begin{verbatim}
./daemon.sh stop
\end{verbatim}

\subsection{General Configuration}\label{sub:generalconfig}
Configuration items that can be changed dynamically are changeable in VELS\_WEB ->
General Config. These configurable items are:
\begin{itemize}
\item {\bf Registration Deadline:} Users who are try to register after this deadline will
    get an error e-mail.
\item {\bf Archive Directory:} Directory in which processed e-mails are moved, this
    directory has to be present on the IMAP server!
\item {\bf Administration E-Mail:} E-Mail addresses which get question and system error E-Mails.

\end{itemize}

\subsection{Whitelisting} \label{sub:whitelisting}
Students that participate in the course have to be whitelisted in the VELS system. If the student
tries to write an E-Mail to the system from an E-Mail address that is not on the Whitelist, an E-Mail
with an error message is sent to him.

Whitelisting can be done in VELS\_WEB under the tab {\it Whitelist}. E-Mail addresses can be added
one at a time or multiple at a time (mass subscription). Removal of a single E-Mail address
can also be done in the VELS\_WEB. Names which will be used when users register can also be 
specified. This is usefull, because many users don't send E-Mails with their name configured
in the "From" header.

\subsection{Configuring the Tasks} \label{sub:configTasks}
Existing tasks can be assembled into a task queue. This configuration is done in VELS\_WEB.
Each task in the queue has to be created with the following properties:
\begin{itemize}
\item {\bf TaskStart:} The start datetime for the task. The task will automatically
    be set to active if this datetime is reached. If auto\_advance is active (configured 
    via config file) users waiting for a task to become active will automatically 
    receive an e-mail with the task description for that task.
\item {\bf TaskDeadline:} The end datetime for the task. Submissions for a tasks after
    this datetime will be rejected.
\item {\bf TaskName:} The name of the folder with the implementation of the task in respect
	to the configured tasks\_dir.
\item {\bf CommonFile:} Name of a common script that offers a tester functionalities.
	Such scripts have to be stored in a directory {\tt ``\_common"} in the configured
	tasks\_dir.
\item {\bf GeneratorExecutable:} The name of the generator executable.
\item {\bf TestExecutable:} The name of the tester executable.
\item {\bf Score:} The score a student gets for successful completion of the task. The
    scores for all completed tasks are added and can therefore also be used for grading.
\item {\bf TaskOperator:} The E-mail(s) of the operator of the task seperated by commas. 
	These task operators are recipient of task specific questions ("Question Task N").
\item {\bf TaskActive:} The state of the task, for inactive tasks the generator won't
    be called. A Task is considered active when the current time is greater than the 
    TaskStart
\end{itemize}

\subsection{Notes on multiple VELS instances on the same machine}

The following should be adhered if you want to use multiple VELS instances on
one machine:
\begin{itemize}
\item Use different users for running the different instances. If you dont't do
	so you migth run into problems concerning the usage of the tmp directories of
	tasks in the test phase.
\item Be sure to use different ports when starting the VELS\_WEB daemon.
\end{itemize}

\subsection{Configuration Checklist} \label{sub:configChecklist}

\begin{enumerate}
\item Installed all needed libraries and tools for autosub, the tasks and VELS\_WEB.
\item Configured the E-Mail server, including an E-Mail archive folder.
\item Created a config file for autosub.
\item Started the autosub system via {\tt autosub.sh start <configfile>}.
\item Started VELS\_WEB using the daemon.
\item Configured all parameters in General Config in VELS\_WEB.
\item Configured all Tasks in VELS\_WEB.
\item Whitelisted all students in VELS\_WEB.
\end{enumerate}

If you forget one of this steps or mis-configure, autosub tries to generate a meaningful
message, still it's nicer to get everything running without being yelled at.

\subsection{The Exam Mode}\label{sub:exammode}
In Exam Mode the students additionally get sent a minimal testbench to test their design.
To enable test mode, change the challenge-mode to exam in VELS\_WEB -> General Config for
an existing course (databases exist) and in the config file for a new course (databases
don't exist).

The remaining configuration is similar to configuration for normal mode.

\subsection{Testing VELS}\label{sub:testingvels}

%QUESTION Martin: Can we write that semester.db and course.db have to be present?,
%some of my tests need them to be present..
% BUMP!!

VELS has a testing suite located in {\tt "autosub/src/tests"}. It consists of unit and
doctests, testing both the autosub system and the tasks itself. With this test suite you
can test if everything is set up the right way before starting a course.To run the test
suite issue the following command from the {\tt "autosub/src/"} directory:

\begin{verbatim}
nosetests3 --with-doctest --doctest-extension=txt --nologcapture -v
\end{verbatim}

This will run all doctest as well as unittest test cases.

If you also want the code coverage of the test-suite then run it as follows:

\begin{verbatim}
nosetests3 --with-doctest --doctest-extension=txt --nologcapture -v --with-coverage
\end{verbatim}

While the above commands run all of the tests, it is also possible to include or exclude only
specific tests. In example it is possible to only execute the load test:

\begin{verbatim}
nosetests3 --nologcapture -s -v  tests/load_test.py:Test_LoadTest
\end{verbatim}

or it is possible to exclude the load test (which makes sense, as that one takes a considerable
amount of time):

\begin{verbatim}
nosetests3 --nologcapture -s -v --ignore-files=load_test.py
\end{verbatim}

The following is tested by the testsuite:
\begin{itemize}
\item Connecting to the databases.
\item Logging a message.
\item Sending an E-Mail.
\item Functionality of the activator thread.
\item Functionality of the generator thread.
\item Functionality of the sender thread.
\item Functionality of the fetcher thread.
\item Functionality of the common used functions.
\item Behavior under high load (load test).
\item Creation off all description files for every task.
\item Compilation of all generated VHDL files with ghdl.
\item Tester functionalities for given right submissions with ghdl.
\end{itemize}

