\section{Specification Overview}
This document aims to specify the tasks and the interaction and processes which are involved 
for the VELS system. Tasks are dynamically generated using a task template and parameters. 
If a student wants a new task the parameters are randomly chosen and inserted into the 
template. The general goal is that every student of a unique course year gets a similar,
but not identical task.

After submission the student's task solution will be checked via a dynamically generated testbench. 
Before testing, the entity is compiled to check for syntax errors. The testbench feeds the student's 
design, the UUT (Unit Under Test) with test vectors and compares the UUT's output to the desired output.
To check the behavior of the UUT as best as possible and prevent the student to design a UUT that 
does only have the right behavior for a limited portion of the test vector space, the test vectors are randomly
chosen from the test vector space. Furthermore every separate submission is checked by a different 
test vectors set, either by permuting, choosing a random portion of the test vector space or by varying the 
test duration. For the case that the testing of a UUT produces an endless loop, each task is tested with a simulation timeout.

To specify each task the following sequential notation is used: 

\begin{description}



%    \item [Number] \textbf{Name} \\
%    \begin{tabular}{|p{2cm}|p{11cm}|}
%        \hline
%        Overview &  
%        \\
%        \hline
%        Description &  
%        \\
%        \hline
%        Creation &  
%        \\
%        \hline
%        Submission &  
%        \\
%        \hline
%        Testing & 
%        \\
%        \hline
%        Feedback & 
%        \\
%        \hline 
%    \end{tabular}

    \item \textbf{<Task Name>}\\
    \begin{tabular}{|p{2cm}|p{11cm}|}
        \hline
        Overview & General Overview of the the task, the intent and learning objective.
        \\
        \hline
        Description & Which informations the student will get to solve the task. 
        \\
        \hline
        Creation & How the individual task for the student is created on the server.  
        \\
        \hline
        Submission & What the student has to submit.
        \\
        \hline
        Testing & How the student solution is tested on the server.
        \\
        \hline
        Feedback & What feedback the student gets after testing.
        \\
        \hline 
    \end{tabular}
\end{description}

A course is composed of a task queue that explicitly orders a selection of the available tasks. Creation
and modification of this queue is done by the VELS Web Interface.


\newpage

All files related to a task are located in a directory. This directory contains:
    
    
\begin{tabular}{|p{3cm}|p{10cm}|}
\hline

Minimal  & \begin{itemize}
    \item {\bf Generator executable:} Generates random parameters for the task, creates entity 
        and behavioral vhdl files and description test and pdf file for the student. Stores the 
        individual task parameters in the \textit{UserTasks} database table.  
    \item {\bf Tester executable:} Given the individual task parameters, tries to compile the 
        student's solution. Generates an individual test bench for the student's solution. Tests 
        the solution and creates feedback text and files for the student. 
    \end{itemize} 
\\
\hline
Optional & \begin{itemize}
    \item {\bf Scripts:} Scripts that are called from the executables in order to aid the 
        generation or test process
    \item {\bf Templates:} These files have to be filled with parameters and can be used to generate entity 
        declarations, test benches or description texts and files for the students.
    \item {\bf Static:} These files are static for the task, therefore the same for every student.
    \item {\bf Exam:} Files that are needed for the VELS exam mode.
\end{itemize} 
\\
\hline
\end{tabular} 


The following sections define a set of initial tasks for the VELS E-learning system. New tasks 
can be added at any time by a course operator via VELS direct server access.
